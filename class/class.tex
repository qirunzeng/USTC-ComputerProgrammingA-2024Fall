\documentclass{beamer}
\usetheme{Madrid} % 可以选择的主题有:AnnArbor, Antibes, Bergen, Berkeley, Berlin, Boadilla, CambridgeUS, Copenhagen, Darmstadt, Dresden, Frankfurt, Goettingen, Hannover, Ilmenau, JuanLesPins, Luebeck, Madrid, Malmoe, Marburg, Montpellier, PaloAlto, Pittsburgh, Rochester, Singapore, Szeged, Warsaw
\usepackage{ctex} % 启用中文支持
\usepackage{amsmath, amssymb, amsthm}
\usepackage{framed}
\usepackage{booktabs}
\usepackage{dsfont}
\usepackage{hyperref}
\hypersetup{colorlinks=true, linkcolor=black}

\title{计算机程序设计习题课}
\author{Qirun Zeng}
\date{\today}

\begin{document}

\begin{frame}
    \titlepage
\end{frame}

\begin{frame}{目录}
    \tableofcontents
\end{frame}


\section{数组}
\begin{frame}{数组的定义}
    \begin{block}{定义}
        数组是一种线性结构,它包含一系列相同类型的元素,这些元素在内存中连续地分布。数组的元素可以通过下标访问。
    \end{block}
    一般的数组定义如下: \\
    \texttt{DataType arrayName[arraySize];} \\
    其中 \texttt{DataType} 为数组的数据类型,\texttt{arrayName} 为数组名,\texttt{arraySize} 为数组的大小。
\end{frame}

\begin{frame}{数组的操作}
    以一维数组为例,数组的操作包括:
    \begin{itemize}
        \item<1-> 初始化数组
        \texttt{DataType arrayName[arraySize] = \{value1, value2, ...\};} \\
        \item<2-> 访问数组元素
        \texttt{arrayName[index];} \\
        \item<3-> 修改数组元素
        \texttt{arrayName[index] = newValue;} \\
        \item<4-> 遍历数组
        \texttt{for (int i = 0; i < arraySize; i++) \{...\}} \\
    \end{itemize}
\end{frame}

\begin{frame}{数组的初始化}
    \begin{block}{初始化}
        数组的初始化可以通过以下几种方式:
        \begin{itemize}
            \item<1-> 指定数组大小,不指定初始值
            \texttt{int a[5];} \\
            \item<2-> 指定数组大小,指定初始值
            \texttt{int a[5] = \{1, 2, 3, 4, 5\};} \\
            \item<3-> 不指定数组大小,指定初始值
            \texttt{int a[] = \{1, 2, 3, 4, 5\};} \\
        \end{itemize}
    \end{block}
\end{frame}

\begin{frame}{数组的遍历}
    \begin{block}{遍历}
        数组的遍历可以通过以下几种方式:
        \begin{itemize}
            \item<1-> 使用下标
            \texttt{for (int i = 0; i < arraySize; i++) \{...\}} \\
            \item<2-> 使用指针
            \texttt{for (int* p = arrayName; p < arrayName + arraySize; p++) \{...\}} \\
        \end{itemize}
    \end{block}
\end{frame}

\begin{frame}{数组的传参}
    \begin{block}{传参}
        数组的传参可以通过以下几种方式:
        \begin{itemize}
            \item<1-> 传递数组名
            \texttt{void func(int a[]) \{...\}} \\
            \item<2-> 传递指针
            \texttt{void func(int* a) \{...\}} \\
        \end{itemize}
    \end{block}
\end{frame}

\section{字符串}
\begin{frame}{字符串的定义}
    \begin{block}{定义}
        字符串是一种特殊的数组,它包含一系列字符,以 \texttt{'\textbackslash0'} 结尾。字符串的元素可以通过下标访问。
    \end{block}
    一般的字符串定义如下: \\
    \texttt{char str[] = "Hello, World!";} \\
    其中 \texttt{char} 为字符类型,\texttt{str} 为字符串名。
\end{frame}

\begin{frame}{字符串的操作}
    以字符串为例,字符串的操作包括:
    \begin{itemize}
        \item<1-> 初始化字符串
        \texttt{char str[] = "Hello, World!";} \\
        \item<2-> 访问字符串元素
        \texttt{str[index];} \\
        \item<3-> 修改字符串元素
        \texttt{str[index] = 'a';} \\
        \item<4-> 遍历字符串
        \texttt{for (int i = 0; str[i] != '\textbackslash0'; i++) \{...\}} \\
    \end{itemize}
\end{frame}

\begin{frame}{字符串的传参}
    \begin{block}{传参}
        字符串的传参可以通过以下几种方式:
        \begin{itemize}
            \item<1-> 传递数组名
            \texttt{void func(char str[]) \{...\}} \\
            \item<2-> 传递指针
            \texttt{void func(char* str) \{...\}} \\
        \end{itemize}
    \end{block}
\end{frame}

\section{结构体}
\begin{frame}{结构体的定义}
    \begin{block}{定义}
        结构体是一种复杂类型,它包含一系列不同类型的元素,这些元素在内存中连续地分布。结构体的元素可以通过成员访问运算符 \texttt{.} 访问。
    \end{block}
    一般的结构体定义如下: \\
    \texttt{struct StructName \{} \\
    \texttt{\ \ \ \ DataType1 member1;} \\
    \texttt{\ \ \ \ DataType2 member2;} \\
    \texttt{\ \ \ \ ...} \\
    \texttt{\};} \\
    \texttt{typedef struct StructName StructName;} \\
    其中 \texttt{DataType} 为结构体的数据类型,\texttt{StructName} 为结构体名。
\end{frame}

\begin{frame}{结构体的操作}
    以结构体为例,结构体的操作包括:
    \begin{itemize}
        \item<1-> 初始化结构体
        \texttt{StructName structName = \{value1, value2, ...\};} \\
        \item<2-> 访问结构体成员
        \texttt{structName.member;} \\
        \item<3-> 修改结构体成员
        \texttt{structName.member = newValue;} \\
    \end{itemize}
\end{frame}

\section{指针}
\begin{frame}{复杂类型的声明}
    \begin{itemize}
        \item 以下声明中,哪些是合法的?如果合法,分别是什么意思?
        \begin{enumerate}[A]
            \item \texttt{int *p;}
            \item \texttt{int a[10];}
            \item \texttt{int *a[10];}
            \item \texttt{int (*a)[10];}
            \item \texttt{int *(*a)[10];}
            \item \texttt{int **a[10];}
            \item \texttt{int **a;}
            \item \texttt{int *a[];}
        \end{enumerate}
    \end{itemize}
\end{frame}

\begin{frame}{解答}
    \begin{enumerate}[A]
        \item \texttt{int *p;} \quad \textcolor{red}{合法},\texttt{p}是一个指向\texttt{int}类型的指针。
        \item \texttt{int a[10];} \quad \textcolor{red}{合法},\texttt{a}是一个长度为10的\texttt{int}数组。
        \item \texttt{int *a[10];} \quad \textcolor{red}{合法},\texttt{a}是一个长度为10的\texttt{int}指针数组。
        \item \texttt{int (*a)[10];} \quad \textcolor{red}{合法},\texttt{a}是一个指向长度为10的\texttt{int}数组的指针。
        \item \texttt{int *(*a)[10];} \quad \textcolor{red}{合法},\texttt{a}是一个指向长度为10的\texttt{int}指针数组的指针。
        \item \texttt{int **a[10];} \quad \textcolor{red}{合法},\texttt{a}是一个长度为10的\texttt{int}指针指针数组。
        \item \texttt{int **a;} \quad \textcolor{red}{合法},\texttt{a}是一个\texttt{int}指针指针。
        \item \texttt{int *a[];} \quad \textcolor{red}{不合法},\texttt{a}是一个未知长度的\texttt{int}指针数组。
    \end{enumerate}
\end{frame}

\begin{frame}{传参的两种方式}
    \texttt{int func1(int HP) \{} \\
    \texttt{\ \ \ \ return HP+1;} \\
    \texttt{\}} \\
    \texttt{int func2(int* HP) \{} \\
    \texttt{\ \ \ \ (*HP)++;} \\
    \texttt{\}} \\
\end{frame}

\begin{frame}{指针的本质}
    指针本质是一 32 位整数或 64 位整数,它存储的是一个内存单元的地址。若以 \textsf{\%d} 形式输出,则与 \texttt{int} 没有区别 \\

    在 C 语言中,指针占据的内存空间是固定的,不受指向的数据类型的影响。一般为 4 字节(32 位系统)或 8 字节(64 位系统)。\\

    
\end{frame}


\section{链表}
\begin{frame}{链表的定义}
    \begin{block}{定义}
        与数组相对,链表是一种链式结构,每个元素都包含一个指向下一个元素的指针。链表的元素可以在内存中不连续地分布,因此可以动态地分配内存。链表的元素称为节点。
    \end{block}
    一般的链表节点定义如下: \\
    \texttt{struct Node \{} \\
    \texttt{\ \ \ \ DataType data;} \\
    \texttt{\ \ \ \ struct Node* next;} \\
    \texttt{\};} \\
    \texttt{typedef struct Node Node, *List;} \\
    其中 \texttt{DataType} 为节点的数据类型,\texttt{next} 为指向下一个节点的指针。

    \texttt{Node} 为节点类型 \texttt{struct Node}\\
    \texttt{List} 为链表类型 \texttt{struct Node*}。
\end{frame}

\begin{frame}{链表的操作}
    以带头节点的单链表为例,链表的操作包括:
    \begin{itemize}
        \item<1-> 创建链表
        \texttt{List createList();} \\
        \item<2-> 插入节点
        \texttt{void insertNode(List L, DataType data);} \\
        \item<3-> 删除节点
        \texttt{void deleteNode(List L, DataType data);} \\
        \item<4-> 遍历链表
        \texttt{void traverseList(List L);} \\
        \item<5-> 释放链表
        \texttt{void freeList(List L);} \\
    \end{itemize}
\end{frame}

\begin{frame}{\texttt{createList()}}
    \texttt{List createList() \{} \\
    \texttt{\ \ \ \ List L = (List)malloc(sizeof(Node));} \\
    \texttt{\ \ \ \ L->next = NULL;} \\
    \texttt{\ \ \ \ return L;} \\
    \texttt{\}} \\
\end{frame}

\begin{frame}{\texttt{insertNode()}}
    \texttt{void insertNode(List L, DataType data) \{} \\
    \texttt{\ \ \ \ List p = L;} \\
    \texttt{\ \ \ \ while (p->next != NULL) p = p->next;} \\
    \texttt{\ \ \ \ List q = (List)malloc(sizeof(Node));} \\
    \texttt{\ \ \ \ q->data = data;} \\
    \texttt{\ \ \ \ q->next = NULL;} \\
    \texttt{\ \ \ \ p->next = q;} \\
    \texttt{\}} \\
\end{frame}

\begin{frame}{\texttt{deleteNode()}}
    \texttt{void deleteNode(List L, DataType data) \{} \\
    \texttt{\ \ \ \ List p = L;} \\
    \texttt{\ \ \ \ while (p->next != NULL \&\& p->next->data != data) p = p->next;} \\
    \texttt{\ \ \ \ if (p->next == NULL) return;} \\
    \texttt{\ \ \ \ List q = p->next;} \\
    \texttt{\ \ \ \ p->next = q->next;} \\
    \texttt{\ \ \ \ free(q);} \\
    \texttt{\}} \\

\end{frame}

\begin{frame}{\texttt{traverseList()}}
    \texttt{void traverseList(List L) \{} \\
    \texttt{\ \ \ \ List p = L->next;} \\
    \texttt{\ \ \ \ while (p != NULL) \{} \\
    \texttt{\ \ \ \ \ \ \ \ printf("\%d ", p->data);} \\
    \texttt{\ \ \ \ \ \ \ \ p = p->next;} \\
    \texttt{\ \ \ \ \}} \\
    \texttt{\}} \\
\end{frame}

\begin{frame}{\texttt{freeList()}}
    \texttt{void freeList(List L) \{} \\
    \texttt{\ \ \ \ List p = L->next;} \\
    \texttt{\ \ \ \ while (p != NULL) \{} \\
    \texttt{\ \ \ \ \ \ \ \ List q = p;} \\
    \texttt{\ \ \ \ \ \ \ \ p = p->next;} \\
    \texttt{\ \ \ \ \ \ \ \ free(q);} \\
    \texttt{\ \ \ \ \}} \\
    \texttt{\ \ \ \ free(L);} \\
    \texttt{\}} \\
\end{frame}



\end{document}